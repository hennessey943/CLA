\documentclass[12pt]{article}

\usepackage{graphics}
\usepackage{amsmath}
\usepackage{amsfonts}
\usepackage{amssymb}
\usepackage[table]{xcolor}
\newcommand\scalemath[2]{\scalebox{#1}{\mbox{\ensuremath{\displaystyle #2}}}}




%\usepackage[active]{srcltx} % SRC Specials for DVI Searching

% Over-full v-boxes on even pages are due to the \v{c} in author's name
\vfuzz2pt % Don't report over-full v-boxes if over-edge is small

% THEOREM Environments ---------------------------------------------------

 \newtheorem{thm}{Theorem}[section]
 \newtheorem{cor}[thm]{Corollary}
 \newtheorem{lem}[thm]{Lemma}
 \newtheorem{prop}[thm]{Proposition}
 %\theoremstyle{definition}
 \newtheorem{defn}[thm]{Definition}
 %\theoremstyle{remark}
 \newtheorem{rem}[thm]{Remark}
 \numberwithin{equation}{section}
% MATH -------------------------------------------------------------------
 \DeclareMathOperator{\RE}{Re}
 \DeclareMathOperator{\IM}{Im}
 \DeclareMathOperator{\ess}{ess}
 \newcommand{\eps}{\varepsilon}
 \newcommand{\To}{\longrightarrow}
 \newcommand{\h}{\mathcal{H}}
 \newcommand{\s}{\mathcal{S}}
 \newcommand{\A}{\mathcal{A}}
 \newcommand{\J}{\mathcal{J}}
 \newcommand{\M}{\mathcal{M}}
 \newcommand{\W}{\mathcal{W}}
 \newcommand{\X}{\mathcal{X}}
 \newcommand{\BOP}{\mathbf{B}}
 \newcommand{\BH}{\mathbf{B}(\mathcal{H})}
 \newcommand{\KH}{\mathcal{K}(\mathcal{H})}
 \newcommand{\Real}{\mathbb{R}}
 \newcommand{\Complex}{\mathbb{C}}
 \newcommand{\Field}{\mathbb{F}}
 \newcommand{\RPlus}{\Real^{+}}
 \newcommand{\Polar}{\mathcal{P}_{\s}}
 \newcommand{\Poly}{\mathcal{P}(E)}
 \newcommand{\EssD}{\mathcal{D}}
 \newcommand{\Lom}{\mathcal{L}}
 \newcommand{\States}{\mathcal{T}}
 \newcommand{\abs}[1]{\left\vert#1\right\vert}
 \newcommand{\set}[1]{\left\{#1\right\}}
 \newcommand{\seq}[1]{\left<#1\right>}
 \newcommand{\norm}[1]{\left\Vert#1\right\Vert}
 \newcommand{\essnorm}[1]{\norm{#1}_{\ess}}
\usepackage{graphicx}
\usepackage{amsmath}
\usepackage{amsfonts}
\usepackage{amssymb}
%TCIDATA{OutputFilter=latex2.dll}
%TCIDATA{CSTFile=LaTeX article (bright).cst}
%TCIDATA{Created=Fri Nov 02 10:44:42 2001}
%TCIDATA{LastRevised=Mon Dec 10 11:56:49 2001}
%TCIDATA{<META NAME="GraphicsSave" CONTENT="32">}
%TCIDATA{<META NAME="DocumentShell" CONTENT="General\Blank Document">}
%TCIDATA{Language=American English}
\newtheorem{theorem}{Theorem}
\newtheorem{acknowledgment}[theorem]{Acknowledgment}
\newtheorem{algorithm}[theorem]{Algorithm}
\newtheorem{axiom}[theorem]{Axiom}
\newtheorem{case}[theorem]{Case}
\newtheorem{claim}[theorem]{Claim}
\newtheorem{conclusion}[theorem]{Conclusion}
\newtheorem{condition}[theorem]{Condition}
\newtheorem{conjecture}[theorem]{Conjecture}
\newtheorem{corollary}[theorem]{Corollary}
\newtheorem{criterion}[theorem]{Criterion}
\newtheorem{definition}[theorem]{Definition}
\newtheorem{example}[theorem]{Example}
\newtheorem{exercise}[theorem]{Exercise}
\newtheorem{lemma}[theorem]{Lemma}
\newtheorem{notation}[theorem]{Notation}
\newtheorem{problem}[theorem]{Problem}
\newtheorem{proposition}[theorem]{Proposition}
\newtheorem{remark}[theorem]{Remark}
\newtheorem{solution}[theorem]{Solution}
\newtheorem{summary}[theorem]{Summary}
\newenvironment{proof}[1][Proof]{\textbf{#1.} }{\ \rule{0.5em}{0.5em}}
\renewcommand\refname{}
\renewcommand\thefootnote{}
\textheight=9in \topmargin=-0.6in \everymath{\displaystyle}
\textwidth=6.5in \oddsidemargin=0.05in
\renewcommand\arraystretch{1.5}
\newenvironment{amatrix}[1]{%
  \left[\begin{array}{@{}*{#1}{c}|c@{}}
}{%
  \end{array}\right]
}
\includeonly{}
\usepackage{amsfonts}
\usepackage{amssymb}
\usepackage{eucal}
\usepackage[bw]{mcode}
\usepackage{listings}
\usepackage{multicol}
\everymath{\displaystyle}
\begin{document}

{\large\bf MATH-6600, CLA Problem Set No. 8, 11-19-15}



\vspace{6 ex}

{\bf Name: Michael Hennessey} \hfill

\vspace{6 ex}

\begin{enumerate}
\item Prove true or false. Throughout, $A\in\mathbb{C}^{m\times m}$ unless otherwise indicated.
    \begin{enumerate}
    \item If $\lambda$ is an eigenvalue of $A$ and $\mu\in\mathbb{C}$, then $\lambda-\mu$ is an eigenvalue of $A-\mu I$.\\

        True.
        $$Ax=\lambda x$$
        $$Ax-\mu x=\lambda x-\mu x$$
        $$(A-\mu I)x=(\lambda -\mu)x$$
        Then $\lambda-\mu$ is an eigenvalue of $A-\mu I$.
    \item If $A$ is real and $\lambda$ is an eigenvalue of $A$, then so is $-\lambda$.\\

    False.
    $$A=\left[\begin{array}{cc}3&0\\0&1\end{array}\right]$$
    $A$ is real with eigenvalues $\lambda_1=3$ and $\lambda_2=1$ and characteristic polynomial
    $$p(\lambda)=(3-\lambda)(1-\lambda).$$
    Clearly $\lambda=-3$ or $\lambda=-1$ do not satisfy the characteristic polynomial.
    \item If $A$ is real and $\lambda$ is an eigenvalue of $a$, then so is $\bar{\lambda}$.\\

    True.
    $$Ax=\lambda x$$
    $$\bar{Ax}=\bar{\lambda x}\implies A\bar{x}=\bar{\lambda}\bar{x}$$
    \item If $\lambda$ is an eigenvalue of $A$ and $A$ is nonsingular, then $\lambda^{-1}$ is an eigenvalue of $A^{-1}$.

        True.\\
        If $A$ is nonsingular, it is diagonalizable. Then
        $$A^{-1}=(X^{-1}\Lambda X)^{-1}=X^{-1}\Lambda^{-1} X.$$
        As $\Lambda$ is the diagonal matrix of eigenvalues, inverting $A$ amounts to inverting the elements of $\Lambda$, the eigenvalues of $A$. Then since the above equation is an eigenvalue decomposition of $A^{-1}$ and $\lambda^{-1}$ is a diagonal entry in $\Lambda^{-1}$, $\lambda^{-1}$ is an eigenvalue of $A^{-1}$.
    \item If all the eigenvalues of $A$ are zero, then $A=0$.\\

    False.\\
    $$A=\left[\begin{array}{cc}0&1\\0&0\end{array}\right]$$
    $$p(\lambda)=\lambda^2\implies \lambda_{1,2}=0$$
    \item If $A$ is hermitian and $\lambda$ is an eigenvalue of $A$, then $|\lambda|$ is a singular value of $A$.\\

        True.\\
        $A^*A=V^*\Sigma^2 V$ has eigenvalues $\sigma_i^2$. $A^*A=A^2$ has eigenvalue decomposition $X^{-1}\Lambda^2 X=V^*\Sigma^2 V$. Then $\lambda^2=\sigma_i^2$. Therefore, the singular values of $A$ are
        $$\sigma_i=|\lambda|.$$

    \item If $A$ is diagonalizable and all its eigenvalues are equal, than $A$ is diagonal.\\

    True.
    $$A=X^{-1}\left[\begin{array}{ccc}\lambda&&\\&\ddots&\\&&\lambda\end{array}\right]\left[\begin{array}{c|c|c}&&\\x_1&\dots&x_m\\&&\end{array}\right]$$
    $$=X^{-1}\left[\begin{array}{c|c|c}&&\\ \lambda x_1&\dots&\lambda x_m\\&&\end{array}\right]=\lambda X^{-1}X=\lambda I$$
    \end{enumerate}
    \pagebreak
\item Let
$$A=\left[\begin{array}{ccc}3&3&0\\0&2&3\\0&0&1\end{array}\right],\hspace{3 ex}B=\left[\begin{array}{ccc}0&0&0\\0&0&0\\\epsilon&0&0\end{array}\right],$$
 with $\epsilon=10^{-10}$.

    \begin{enumerate}
    \item Estimate the locations of the eigenvalues of $A+B$ by using Gershgorin's theorem.\\

    Solution:\\

    $$A+B=\left[\begin{array}{ccc}3&3&0\\0&2&3\\ \epsilon&0&1\end{array}\right]$$
    $$D_1:\{z|\hspace{1 ex} |z-3|\leq 3\}$$
    $$D_2:\{z|\hspace{1 ex} |z-2|\leq 3\}$$
    $$D_3:\{z|\hspace{1 ex} |z-1|\leq \epsilon\}$$
    Where $D_i$ corresponds to the $i$th row of $A+B$. Therefore $\lambda_1\in D_1$, $\lambda_2\in D_2$, and $\lambda_3\in D_3$ by Gershgorin's Circle theorem.

    \item Improve the estimates from (a) by judicious choices of diagonal similarity transformations of the form
        $$D=\left[\begin{array}{ccc}1&0&0\\0&d&0\\0&0&d^2\end{array}\right].$$

    Solution:\\

    We begin by doing a diagonal similarity transform of $A+B$ to get
    $$D^{-1}(A+B)D=\left[\begin{array}{ccc}1&0&0\\0&1/d&0\\0&0&1/d^2\end{array}\right]\left[\begin{array}{ccc}3&3&0\\0&2&3\\ \epsilon &0&1\end{array}\right]\left[\begin{array}{ccc}1&0&0\\0&d&0\\0&0&d^2\end{array}\right]=\left[\begin{array}{ccc}3&3d&0\\0&2&3d\\ \epsilon/d^2&0&1\end{array}\right]$$
    Thus we have
    $$D^{-1}(A+B)D=\left[\begin{array}{ccc}3&&\\&2&\\&&1\end{array}\right]+\left[\begin{array}{ccc}0&3d&0\\0&0&3d\\ \epsilon/d^2&0&0\end{array}\right].$$
    This gives the three Gershgorin disks:
    $$D_1:\{z|\hspace{1 ex} |z-3|\leq 3d\}$$
    $$D_2:\{z|\hspace{1 ex} |z-2|\leq 3d\}$$
    $$D_3:\{z|\hspace{1 ex} |z-1|\leq \frac{\epsilon}{d^2}\}$$
    where $D_i$ corresponds to the $i$th row of $D^{-1}(A+B)D$. Thus $\lambda_1\in D_1$, $\lambda_2\in D_2$, and $\lambda_3\in D_3$ by Gershgorin's Circle theorem. To "zoom" in on each eigenvalue we can choose appropriate values of $d$ such that the radius is as small as possible around each eigenvalue and the disk in question has no intersection with the other two disks.\\
    We begin with $D_1$. Here we want to zoom in on $|z-3|$ and stop $D_2$ from intersecting $D_1$. We must also be careful to not make $d$ so large that $D_3$ begins intersecting $D_1$. To choose $d$ such that these conditions are satisfied we expand the absolute value inequalities that define the disks:
    $$D_1: 3-3d\leq z\leq 3+3d$$
    $$D_2:2-3d\leq z\leq 2+3d$$
    $$D_3:1-\frac{\epsilon}{d^2}\leq z\leq 1+\frac{\epsilon}{d^2}$$
    Since we want $d$ to be as small as possible, and since $D_2$ will not be an issue, the only constraint we must keep in mind is:
    $$ 1+\frac{\epsilon}{d^2}<3+3d$$
    To simplify the analysis we let $d=\frac{\sqrt{\epsilon}}{a}$. We then have
    $$1+a^2<3+3\frac{\sqrt{\epsilon}}{a}\implies a^2\approx 2$$
    For clarity, we choose $a=\sqrt{1.8}=\frac{3}{\sqrt{5}}$, which makes $d=\frac{\sqrt{5\epsilon}}{3}$. Then we know $\lambda_1$ is in the Gershgorin disk
    $$D_1':\{z:|z-3| \leq \sqrt{5\epsilon}\}.$$
    Moving on to $D_2$ we use the same inequalities as above, except our constraints are that we need
    $$2+3d<3-3d\text{  and  }1+\frac{\epsilon}{d^2}<2-3d.$$
    To satisfy the first constraint all we need is $d<\frac{1}{6}.$ To satisfy the second, however, we can make a similar substitution as before, with $d=\frac{\sqrt{\epsilon}}{b}$. The second constraint then becomes:
    $$1+b^2<2-\frac{3}{b}\sqrt{\epsilon}\implies b^2<1.$$
    Again, for clarity, we choose $b=\sqrt{.9}=\frac{3}{\sqrt{10}}$, which makes $d=\frac{\sqrt{10\epsilon}}{3}$. Then we know that $\lambda_2$ lies in the Gershgorin disk
    $$D_2':\{z:|z-2|\leq\sqrt{10\epsilon}\}.$$
    Lastly, we look at $D_3.$ Here we are only concerned with removing the overlap from $D_2$ and $D_1$. Using the same equalities as before, we note that if $D_2\cap D_3=\emptyset$ then $D_1\cap D_3=\emptyset$. Therefore, we are only worried about the constraint
    $$ 1+\frac{\epsilon}{d^2}<2-3d.$$
    Except here we are only interested in making $d$ small enough so that $D_2$ does not intersect $D_3$. We can easily see that we want $d<\frac{1}{3}$. For a very good estimate, we can let $d=.3=\frac{3}{10}$. Then we know $\lambda_3$ is in the Gershgorin disk
    $$D_3'=\{z:|z-1|\leq \frac{100}{9}\epsilon\}.$$
    \end{enumerate}

    \item \begin{enumerate}
        \item Let $A\in\mathbb{C}^{m\times m}$ be tridiagonal and hermitian, with all its sub- and superdiagonal entries nonzero. Prove that the eigenvalues of $A$ are distinct.\\

            Proof: Let $\lambda$ be an eigenvalue of $A$. Then $B=A-\lambda I$ is singular, and therefore has rank$(A-\lambda I)\leq m-1$. We then note the $m-1\times m$ submatrix $B_{2:m,1:m}$ is upper triangular whose diagonal entries are non-zero by our assumptions on $A$. Hence $B_{2:m,1:m}$ has $m-1$ linearly independent columns and is therefore of full rank. Therefore, $A-\lambda I$ must have rank $m-1$. Therefore, the null space of $B$ is spanned by one unique eigenvector of $A$ corresponding to $\lambda$. Since $A$ is Hermitian, it is nonsingular and therefore has $m$ linearly independent eigenvectors, by the finite-dimensional spectral theorem. Thus all $\lambda$ must be distinct.

        \item On the other hand, let $A$ be upper-Hessenberg, with all its subdiagonal entries nonzero. Give an example that shows that the eigenvalues of $A$ are not necessarily distinct.\\

            Solution:
            $$A=\left[\begin{array}{ccc}0&0&0\\2&0&0\\0&2&0\end{array}\right]$$
            has eigenvalues all 0.
        \end{enumerate}
\pagebreak
    \item Write a Matlab code \mcode{[W,H] = hessenberg(A)} to transform an $m\times m$ matrix $A$ to upper Hessenberg form, $H$, by similarity transformations using Householder reflectors,
        $$A=QHQ^*.$$
        Also write a Matlab function \mcode{[Q] = formQh(A)} that takes $W$ and generates the matrix $Q$.\\
        Test your routine on the $m\times m$ matrix $A=[a_{ij}]$ with entries
        $$a_{ij}=9, \hspace{2 ex}\text{for }i=j,$$
        $$a_{ij}=\frac{1}{i+j}\hspace{1 ex}\text{ for }i\neq j$$
        and $m=5.$
        Output $A,H,W,Q,\norm{Q^*Q-I}_2$, and $\norm{A-QHQ^*}_2.$\\
        
        Code:
        \begin{lstlisting}
        function [W,H]=hessenberg(A)

[~,m]=size(A);
for k=1:m-2
    I=(k+1):m;
    x=A(k+1:m,k);
    e=zeros(m-k,1);
    e(1)=1;
    if x(1)==0
        V(I,k)=norm(x,2)*e+x;
    else
        V(I,k)=sign(x(1))*norm(x,2)*e+x;
    end
    V(I,k)=V(I,k)/norm(V(I,k),2);
    A(k+1:m,k:m)=A(k+1:m,k:m)-2*V(I,k)*(V(I,k)'*A(k+1:m,k:m));
    A(1:m,k+1:m)=A(1:m,k+1:m)-2*(A(1:m,k+1:m)*V(I,k))*V(I,k)';
end
W=V;
H=A(1:m,1:m);
\end{lstlisting}
\begin{lstlisting}
function Q=formQh(W)

[m,~]=size(W);

for i=1:m
    x=zeros(m,1);
    x(i)=1;
    for k=m-2:-1:1
        x(k:m)=x(k:m)-2*W(k:m,k)*(W(k:m,k)'*x(k:m));

    end
    Q(1:m,i)=x;
end
\end{lstlisting}
\begin{lstlisting}
%hessenberg matrix precursor
m=5;
A=zeros(m,m);
for i=1:m
    for j=1:m
        if i==j
            A(i,j)=9;
        else
            A(i,j)=1/(i+j);
        end
    end
end
\end{lstlisting}
Results:
\begin{lstlisting}
>> hessenbergprecursor
>> [W,H]=hessenberg(A);
>> Q=formQh(W);
>> A

A =

    9.0000    0.3333    0.2500    0.2000    0.1667
    0.3333    9.0000    0.2000    0.1667    0.1429
    0.2500    0.2000    9.0000    0.1429    0.1250
    0.2000    0.1667    0.1429    9.0000    0.1111
    0.1667    0.1429    0.1250    0.1111    9.0000

>> H

H =

    9.0000   -0.4913         0         0         0
   -0.4913    9.4289    0.1080   -0.0000    0.0000
         0    0.1080    8.8463    0.0400    0.0000
         0         0    0.0400    8.8507   -0.0208
         0   -0.0000    0.0000   -0.0208    8.8741

>> W

W =

         0         0         0
    0.9161         0         0
    0.2777   -0.8281         0
    0.2222   -0.3721   -0.7992
    0.1851   -0.4194   -0.6010

>> Q

Q =

    1.0000         0         0         0         0
         0   -0.6785    0.6754   -0.2850   -0.0479
         0   -0.5088   -0.1666    0.7519    0.3847
         0   -0.4071   -0.4524    0.0300   -0.7929
         0   -0.3392   -0.5580   -0.5938    0.4701

>> norm(Q'*Q-eye(5),2)

ans =

   9.7450e-16

>> norm(A-Q*H*Q',2)

ans =

   1.8432e-14
\end{lstlisting}
 \end{enumerate}
 \end{document}