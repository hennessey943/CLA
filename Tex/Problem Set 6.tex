\documentclass[12pt]{article}

\usepackage{graphics}
\usepackage{amsmath}
\usepackage{amsfonts}
\usepackage{amssymb}
\usepackage[table]{xcolor}



%\usepackage[active]{srcltx} % SRC Specials for DVI Searching

% Over-full v-boxes on even pages are due to the \v{c} in author's name
\vfuzz2pt % Don't report over-full v-boxes if over-edge is small

% THEOREM Environments ---------------------------------------------------

 \newtheorem{thm}{Theorem}[section]
 \newtheorem{cor}[thm]{Corollary}
 \newtheorem{lem}[thm]{Lemma}
 \newtheorem{prop}[thm]{Proposition}
 %\theoremstyle{definition}
 \newtheorem{defn}[thm]{Definition}
 %\theoremstyle{remark}
 \newtheorem{rem}[thm]{Remark}
 \numberwithin{equation}{section}
% MATH -------------------------------------------------------------------
 \DeclareMathOperator{\RE}{Re}
 \DeclareMathOperator{\IM}{Im}
 \DeclareMathOperator{\ess}{ess}
 \newcommand{\eps}{\varepsilon}
 \newcommand{\To}{\longrightarrow}
 \newcommand{\h}{\mathcal{H}}
 \newcommand{\s}{\mathcal{S}}
 \newcommand{\A}{\mathcal{A}}
 \newcommand{\J}{\mathcal{J}}
 \newcommand{\M}{\mathcal{M}}
 \newcommand{\W}{\mathcal{W}}
 \newcommand{\X}{\mathcal{X}}
 \newcommand{\BOP}{\mathbf{B}}
 \newcommand{\BH}{\mathbf{B}(\mathcal{H})}
 \newcommand{\KH}{\mathcal{K}(\mathcal{H})}
 \newcommand{\Real}{\mathbb{R}}
 \newcommand{\Complex}{\mathbb{C}}
 \newcommand{\Field}{\mathbb{F}}
 \newcommand{\RPlus}{\Real^{+}}
 \newcommand{\Polar}{\mathcal{P}_{\s}}
 \newcommand{\Poly}{\mathcal{P}(E)}
 \newcommand{\EssD}{\mathcal{D}}
 \newcommand{\Lom}{\mathcal{L}}
 \newcommand{\States}{\mathcal{T}}
 \newcommand{\abs}[1]{\left\vert#1\right\vert}
 \newcommand{\set}[1]{\left\{#1\right\}}
 \newcommand{\seq}[1]{\left<#1\right>}
 \newcommand{\norm}[1]{\left\Vert#1\right\Vert}
 \newcommand{\essnorm}[1]{\norm{#1}_{\ess}}
\usepackage{graphicx}
\usepackage{amsmath}
\usepackage{amsfonts}
\usepackage{amssymb}
%TCIDATA{OutputFilter=latex2.dll}
%TCIDATA{CSTFile=LaTeX article (bright).cst}
%TCIDATA{Created=Fri Nov 02 10:44:42 2001}
%TCIDATA{LastRevised=Mon Dec 10 11:56:49 2001}
%TCIDATA{<META NAME="GraphicsSave" CONTENT="32">}
%TCIDATA{<META NAME="DocumentShell" CONTENT="General\Blank Document">}
%TCIDATA{Language=American English}
\newtheorem{theorem}{Theorem}
\newtheorem{acknowledgment}[theorem]{Acknowledgment}
\newtheorem{algorithm}[theorem]{Algorithm}
\newtheorem{axiom}[theorem]{Axiom}
\newtheorem{case}[theorem]{Case}
\newtheorem{claim}[theorem]{Claim}
\newtheorem{conclusion}[theorem]{Conclusion}
\newtheorem{condition}[theorem]{Condition}
\newtheorem{conjecture}[theorem]{Conjecture}
\newtheorem{corollary}[theorem]{Corollary}
\newtheorem{criterion}[theorem]{Criterion}
\newtheorem{definition}[theorem]{Definition}
\newtheorem{example}[theorem]{Example}
\newtheorem{exercise}[theorem]{Exercise}
\newtheorem{lemma}[theorem]{Lemma}
\newtheorem{notation}[theorem]{Notation}
\newtheorem{problem}[theorem]{Problem}
\newtheorem{proposition}[theorem]{Proposition}
\newtheorem{remark}[theorem]{Remark}
\newtheorem{solution}[theorem]{Solution}
\newtheorem{summary}[theorem]{Summary}
\newenvironment{proof}[1][Proof]{\textbf{#1.} }{\ \rule{0.5em}{0.5em}}
\renewcommand\refname{}
\renewcommand\thefootnote{}
\textheight=9in \topmargin=-0.6in \everymath{\displaystyle}
\textwidth=6.5in \oddsidemargin=0.05in
\renewcommand\arraystretch{1.5}
\newenvironment{amatrix}[1]{%
  \left[\begin{array}{@{}*{#1}{c}|c@{}}
}{%
  \end{array}\right]
}
\includeonly{}
\usepackage{amsfonts}
\usepackage{amssymb}
\usepackage{eucal}
\everymath{\displaystyle}
\begin{document}

{\large\bf MATH-6800, CLA: Problem Set 6, 11-5-15}



\vspace{6 ex}

{\bf Name: Michael Hennessey} \hfill

\vspace{6 ex}

\begin{enumerate}
\item Given $A\in \mathbb{C}^{m\times n}$ of rank $n$ nd $b\in\mathbb{C}^m$, consider the block $2\times 2$ system of equations
    $$\left[\begin{array}{cc}I&A\\A^*&0\end{array}\right]\left[\begin{array}{c}r\\x\end{array}\right]=\left[\begin{array}{c}b\\0\end{array}\right],$$
    where $I$ is the $m\times m$ identity. Show that this system has a unique solution $(r,x)^T$, and that the vectors $r$ and $x$ are the residual and the solution of the least squares problem. \\

    Solution:\\

    We begin by using block-multiplication to rewrite the equation:
    $$\left[\begin{array}{c}r+Ax\\A^*r\end{array}\right]=\left[\begin{array}{c}b\\0\end{array}\right].$$

    This gives the system of equations:
    $$\begin{array}{c}r+Ax=b\\A^*r=0\end{array}.$$

    We combine the two equations by multiplying the first by $A^*$ to get
    $$A^*r+A^*Ax=A^*b\implies A^*Ax=A^*b.$$

    By Theorem 11.1 $x$ and $r$ are the solution and residual to the least squares problem, since
    $$A^*r=0\text{  and  }A^*Ax=A^*b.$$

    By the same theorem, as $A$ has full rank, the solution $x$ and therefore the residual $r$ are uniquely determined.

\item Let $A\in\mathbb{C}^{m\times m}$ be nonsinguler. Show that $A$ has an $LU$ factorization if and only if for each $k$ with $1\leq k\leq m$, the upper left $k\times k$ block $A_{1:k,1:k}$ is nonsingular. Prove that this $LU$ factorization is unique.\\
    
    Solution:\\
    
    $\Rightarrow$ If the upper left $k\times k$ block of $A$ is nonsingular, it's rows and columns are linearly independent. Therefore if we were to attempt to factor the upper left block of $A$ using Gaussian Elimination, the diagonal entries of the block will remain or become nonzero upon eliminating the sub diagonal entries in the subtraction step of the GE algorithm without pivoting. This is due to the fact that the block will become upper triangular and have a nonzero determinant after the elimination steps. Therefore, as the $LU$ factorization will complete without pivoting for each $k<m$ inductively, it will complete for the entire matrix. \\ 
    $\Leftarrow$ If $A$ has an $LU$ factorization, $A=LU$, then we can write it block form:
    $$A=\left[\begin{array}{cc}A_{11}&A_{12}\\A_{21}&A_{22}\end{array}\right]=\left[\begin{array}{cc}L_{11}&0\\L_{21}&L_{22}\end{array}\right]\left[\begin{array}{cc}U_{11}&U_{12}\\0&U_{22}\end{array}\right].$$
    
    This implies for any sized $A_{11}$ the upper left block of $A$ has an $LU$ factorization 
    $$A_{11}=L_{11}U_{11}.$$
    Then $\det(A_{11})=\det(L_{11})\det(U_{11})\neq 0$.\\
    
    To prove that the factorization is unique, we assume it is not: $A=L_1U_1=L_2U_2$. 
    Then through some manipulation
    $$L_2^{-1}L_1=U_2U_1^{-1}$$
    As the inverse of a lower triangular matrix is lower triangular and as the inverse of an upper triangular matrix is upper triangular, the product $L_2^{-1}L_1$ is lower triangular and the product $U_2U_1^{-1}$ is upper triangular. Then, the only way the above equation is true is if  
    $$L_2^{-1}L_1=U_2U_1^{-1}=I.$$
    
    Therefore the $LU$ factorization of $A$ is unique.

\item Suppose $A\in\mathbb{C}^{m\times m}$ is nonsingular and has a unique $LU$ factorization. Let $A$ be banded with bandwidth $2p+1$, i.e., $a_{ij}=0$ for $\abs{i-j}>p.$ What can you say about the sparsity patterns of $L$ and $U$.\\
    
    Solution:\\
    
    $L$ is lower triangular with $p$ nonzero entries directly below the diagonal in each column. $U$ is upper triangular with $p$ nonzero entries directly above the diagonal in each column.
    
    $$a_{ij}=\sum_{k=1}^j l_{ik}u_{kj}=0 \text{ for } \abs{i-j}>p$$
    When $i-j>p$, $l_{ij}=0$ and when $j-i>p$, $u_{ij}=0$.
    
\item Let $A$ be the $4 \times 4$ matrix 
$$A=\left[\begin{array}{cccc} 2&1&1&0\\4&3&3&1\\8&7&9&5\\6&7&9&8\end{array}\right].$$
    \begin{enumerate}
    \item Determine $\det A$ from the $LU$ factorization without pivoting.\\
    
    Solution:\\
    
    $$A=LU=\left[\begin{array}{cccc}1&0&0&0\\2&1&0&0\\4&3&1&0\\3&4&1&1\end{array}\right]\left[\begin{array}{cccc}2&1&1&0\\0&1&1&1\\0&0&2&2\\0&0&0&2\end{array}\right]$$
    Then,
    $$\det(A)=\det(LU)=\det(L)\det(U)=(1)(2)(2)(2)=8.$$
    
    \item Determine $\det(A)$ from the $LU$ factorization of $A$ with pivoting.\\
    
    Solution:\\
    
    $$PA=LU\implies$$ $$\left[\begin{array}{cccc}0&0&1&0\\0&0&0&1\\0&1&0&0\\1&0&0&0\end{array}\right]\left[\begin{array}{cccc}2&1&1&0\\4&3&3&1\\8&7&9&5\\6&7&9&8\end{array}\right]=\left[\begin{array}{cccc}1&0&0&0\\3/4&1&0&0\\1/2&-2/7&1&0\\1/4&-3/7&1/3&1\end{array}\right]\left[\begin{array}{cccc}8&7&9&5\\0&7/4&9/4&17/4\\0&0&-6/7&-2/7\\0&0&0&2/3\end{array}\right]$$
    
    Then,
    
    $$\det(PA)=\det(LU)\implies \det(P)\det(A)=\det(L)\det(U)$$
    $$\implies -\det(A)=(8)(7/4)(-6/7)(2/3)\implies \det(A)=8.$$
    
    \item Describe how Gaussian Elimination with partial pivoting can be used to find the determinant of a general square matrix.\\
        
        Solution:\\
        
        As GE with partial pivoting will result in a factorization $PA=LU$ where $P$ is a permutation matrix, $L$ is lower triangular and $U$ is upper triangular. Then 
        $$\det(A)=\frac{\det(L)\det(U)}{\det(P)}=\frac{\prod_{j=1}^m l_{jj}\prod_{i=1}^m u_{ii}}{(-1)^n}$$
        where $n$ is the number of permutations of $I$ required to produce $P$.
        \end{enumerate}
        
\item Consider Gaussian elimination carried out with pivoting by columns instead of rows, leading to a factorization $AQ=LU$, where $Q$ is a permutation matrix.
    \begin{enumerate}
    \item Show that if $A$ is nonsingular, such a factorization always exists.\\
    
    Solution:\\
    
    Since $A$ is nonsingular it must have at least one nonzero element in the first row. We use a permutation matrix $Q_1$ to get $AQ_1$ to have the nonzero element in the first position $a_{11}\neq 0$. Then we can carry out Gaussian elimination on the first column, giving 
    $A=L_1AQ_1$. Using the fact that $A$ is nonsingular again, the $m-1\times m-1$ block in the lower right hand corner of the new matrix must also be nonsingular. We repeat the shifting and Gaussian elimination steps to completion using the same procedure and logic.
    
    \item Show that if $A$ is singular, such a factorization does not always exist.\\
    
    Solution:\\
    
    Any matrix whose first row is a row of zeros can not be factored in this manner.
    
    \end{enumerate}
    
\end{enumerate}
\end{document} 