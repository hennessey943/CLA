\documentclass[12pt]{article}

\usepackage{graphics}
\usepackage{amsmath}
\usepackage{amsfonts}
\usepackage{amssymb}
\usepackage[table]{xcolor}
\newcommand\scalemath[2]{\scalebox{#1}{\mbox{\ensuremath{\displaystyle #2}}}}




%\usepackage[active]{srcltx} % SRC Specials for DVI Searching

% Over-full v-boxes on even pages are due to the \v{c} in author's name
\vfuzz2pt % Don't report over-full v-boxes if over-edge is small

% THEOREM Environments ---------------------------------------------------

 \newtheorem{thm}{Theorem}[section]
 \newtheorem{cor}[thm]{Corollary}
 \newtheorem{lem}[thm]{Lemma}
 \newtheorem{prop}[thm]{Proposition}
 %\theoremstyle{definition}
 \newtheorem{defn}[thm]{Definition}
 %\theoremstyle{remark}
 \newtheorem{rem}[thm]{Remark}
 \numberwithin{equation}{section}
% MATH -------------------------------------------------------------------
 \DeclareMathOperator{\RE}{Re}
 \DeclareMathOperator{\IM}{Im}
 \DeclareMathOperator{\ess}{ess}
 \newcommand{\eps}{\varepsilon}
 \newcommand{\To}{\longrightarrow}
 \newcommand{\h}{\mathcal{H}}
 \newcommand{\s}{\mathcal{S}}
 \newcommand{\A}{\mathcal{A}}
 \newcommand{\J}{\mathcal{J}}
 \newcommand{\M}{\mathcal{M}}
 \newcommand{\W}{\mathcal{W}}
 \newcommand{\X}{\mathcal{X}}
 \newcommand{\BOP}{\mathbf{B}}
 \newcommand{\BH}{\mathbf{B}(\mathcal{H})}
 \newcommand{\KH}{\mathcal{K}(\mathcal{H})}
 \newcommand{\Real}{\mathbb{R}}
 \newcommand{\Complex}{\mathbb{C}}
 \newcommand{\Field}{\mathbb{F}}
 \newcommand{\RPlus}{\Real^{+}}
 \newcommand{\Polar}{\mathcal{P}_{\s}}
 \newcommand{\Poly}{\mathcal{P}(E)}
 \newcommand{\EssD}{\mathcal{D}}
 \newcommand{\Lom}{\mathcal{L}}
 \newcommand{\States}{\mathcal{T}}
 \newcommand{\abs}[1]{\left\vert#1\right\vert}
 \newcommand{\set}[1]{\left\{#1\right\}}
 \newcommand{\seq}[1]{\left<#1\right>}
 \newcommand{\norm}[1]{\left\Vert#1\right\Vert}
 \newcommand{\essnorm}[1]{\norm{#1}_{\ess}}
\usepackage{graphicx}
\usepackage{amsmath}
\usepackage{amsfonts}
\usepackage{amssymb}
%TCIDATA{OutputFilter=latex2.dll}
%TCIDATA{CSTFile=LaTeX article (bright).cst}
%TCIDATA{Created=Fri Nov 02 10:44:42 2001}
%TCIDATA{LastRevised=Mon Dec 10 11:56:49 2001}
%TCIDATA{<META NAME="GraphicsSave" CONTENT="32">}
%TCIDATA{<META NAME="DocumentShell" CONTENT="General\Blank Document">}
%TCIDATA{Language=American English}
\newtheorem{theorem}{Theorem}
\newtheorem{acknowledgment}[theorem]{Acknowledgment}
\newtheorem{algorithm}[theorem]{Algorithm}
\newtheorem{axiom}[theorem]{Axiom}
\newtheorem{case}[theorem]{Case}
\newtheorem{claim}[theorem]{Claim}
\newtheorem{conclusion}[theorem]{Conclusion}
\newtheorem{condition}[theorem]{Condition}
\newtheorem{conjecture}[theorem]{Conjecture}
\newtheorem{corollary}[theorem]{Corollary}
\newtheorem{criterion}[theorem]{Criterion}
\newtheorem{definition}[theorem]{Definition}
\newtheorem{example}[theorem]{Example}
\newtheorem{exercise}[theorem]{Exercise}
\newtheorem{lemma}[theorem]{Lemma}
\newtheorem{notation}[theorem]{Notation}
\newtheorem{problem}[theorem]{Problem}
\newtheorem{proposition}[theorem]{Proposition}
\newtheorem{remark}[theorem]{Remark}
\newtheorem{solution}[theorem]{Solution}
\newtheorem{summary}[theorem]{Summary}
\newenvironment{proof}[1][Proof]{\textbf{#1.} }{\ \rule{0.5em}{0.5em}}
\renewcommand\refname{}
\renewcommand\thefootnote{}
\textheight=9in \topmargin=-0.6in \everymath{\displaystyle}
\textwidth=6.5in \oddsidemargin=0.05in
\renewcommand\arraystretch{1.5}
\newenvironment{amatrix}[1]{%
  \left[\begin{array}{@{}*{#1}{c}|c@{}}
}{%
  \end{array}\right]
}
\includeonly{}
\usepackage{amsfonts}
\usepackage{amssymb}
\usepackage{eucal}
\usepackage[bw]{mcode}
\usepackage{listings}
\usepackage{multicol}
\everymath{\displaystyle}
\begin{document}

{\large\bf MATH-6600, CLA Problem Set No. 7, 11-12-15}



\vspace{6 ex}

{\bf Name: Michael Hennessey} \hfill

\vspace{6 ex}

\begin{enumerate}
\item Derive a block LU decomposition assuming no pivoting of a block tridiagonal matrix $D$. What are the conditions you need for this LU decomposition to exist?\\

Solution:\\

We can essentially use the Gaussian Elimination (GE) algorithm without pivoting to compute the block LU factorization. We can begin by eliminating $A_2$:
$$\left[\begin{array}{ccccc}I&&&&\\F&I&&&\\&&I&&\\&&&\ddots&\\&&&&I\end{array}\right]\left[\begin{array}{ccccc}B_1&C_1&&&\\A_2&B_2&C_2&&\\&A_3&B_3&C_3&\\&&\ddots&\ddots&\ddots\\&&&A_n&B_n\end{array}\right]=\left[\begin{array}{ccccc}B_1&C_1&&&\\0&B_2+FC_1&C_2&&\\&A_3&B_3&C_3&\\&&\ddots&\ddots&\ddots\\&&&A_n&B_n\end{array}\right]$$
Then
$$FB_1+A_2=0\implies F=-A_2B_1^{-1}.$$
This gives the first step of the block LU factorization:
$$L_1^{-1}D=U_1\implies$$
$$\left[\begin{array}{ccccc}I&&&&\\-A_2B_1^{-1}&I&&&\\&&I&&\\&&&\ddots&\\&&&&I\end{array}\right]\left[\begin{array}{ccccc}B_1&C_1&&&\\A_2&B_2&C_2&&\\&A_3&B_3&C_3&\\&&\ddots&\ddots&\ddots\\&&&A_n&B_n\end{array}\right]=\left[\begin{array}{ccccc}B_1&C_1&&&\\0&B_2-A_2B_1^{-1}C_1&C_2&&\\&A_3&B_3&C_3&\\&&\ddots&\ddots&\ddots\\&&&A_n&B_n\end{array}\right].$$

We can multiply both sides by
$$L_1=\left[\begin{array}{ccccc}I&&&&\\A_2B_1^{-1}&I&&&\\&&I&&\\&&&\ddots&\\&&&&I\end{array}\right]$$
to get the $D=LU$ form, just as in the non-block case.
$$\left[\begin{array}{ccccc}B_1&C_1&&&\\A_2&B_2&C_2&&\\&A_3&B_3&C_3&\\&&\ddots&\ddots&\ddots\\&&&A_n&B_n\end{array}\right]=\left[\begin{array}{ccccc}I&&&&\\A_2B_1^{-1}&I&&&\\&&I&&\\&&&\ddots&\\&&&&I\end{array}\right]\left[\begin{array}{ccccc}B_1&C_1&&&\\0&B_2-A_2B_1^{-1}C_1&C_2&&\\&A_3&B_3&C_3&\\&&\ddots&\ddots&\ddots\\&&&A_n&B_n\end{array}\right]$$
Then continuing on in this fashion will result in the block LU decomposition:
\[\scalemath{0.8}{\left[\begin{array}{ccccc}B_1&C_1&&&\\A_2&B_2&C_2&&\\&A_3&B_3&C_3&\\&&\ddots&\ddots&\ddots\\&&&A_n&B_n\end{array}\right]=\left[\begin{array}{ccccc}I&&&&\\A_2B_1^{-1}&I&&&\\&(B_2-A_2B_1^{-1}C_1)^{-1}&I&&\\&&\ddots&\ddots&\\&&&\tilde{B}_n&I\end{array}\right]\left[\begin{array}{ccccc}B_1&C_1&&&\\0&B_2-A_2B_1^{-1}C_1&C_2&&\\&0&\tilde{B}_3&C_3&\\&&\ddots&\ddots&\ddots\\&&&&\tilde{B}_n\end{array}\right]}\]
where $\tilde{B}_j$ is defined thus:
$$\tilde{B}_j=B_j-A_j(B_{j-1}-A_{j-1}(B_{j-2}-...-A_2B_1^{-1}C_1)^{-1}C_2)^{-1}...C_{j-2})^{-1}C_{j-1}.$$
Therefore, for the block LU decomposition to exist, the diagonal elements $\tilde{B_j}$ must be invertible for $j=1,...,n-1$.\\

\item Suppose an $m\times m$ matrix $A$ is written in the block form  $A=\left[\begin{array}{cc}A_{11}&A_{12}\\A_{21}&A_{22}\end{array}\right]$, where $A_{11}$ is $n\times n$ and $A_{22}$ is $(m-n)\times(m-n).$\\
Assume that $A$ is nonsingular and has a unique LU factorization.\\
\begin{enumerate}
\item Verify the formula
$$\left[\begin{array}{cc}I& \\-A_{21}A_{11}^{-1}&I\end{array}\right]\left[\begin{array}{cc}A_{11}&A_{12}\\A_{21}&A_{22}\end{array}\right]=\left[\begin{array}{cc}A_{11}&A_{12}\\&A_{22}-A_{21}A_{11}^{-1}A_{12}\end{array}\right]$$
for "elimination" of the block $A_{21}$. The matrix $A_{22}-A_{21}A_{11}^{-1}A_{12}$ is known as the Schur complement of $A_{11}$ in $A$.\\

Solution:\\

Multiplying the left hand side of the equation using block-multiplication readily verifies the formula:
$$\left[\begin{array}{cc}I& \\-A_{21}A_{11}^{-1}&I\end{array}\right]\left[\begin{array}{cc}A_{11}&A_{12}\\A_{21}&A_{22}\end{array}\right]=\left[\begin{array}{cc}A_{11}&A_{12}\\-A_{21}A_{11}^{-1}A_{11}+A_{21}&-A_{21}A_{11}^{-1}A_{12}+A_{22}\end{array}\right]$$
$$=\left[\begin{array}{cc}A_{11}&A_{12}\\-A_{21}+A_{21}&A_{22}-A_{21}A_{11}^{-1}A_{12}\end{array}\right]=\left[\begin{array}{cc}A_{11}&A_{12}\\&A_{22}-A_{21}A_{11}^{-1}A_{12}\end{array}\right]$$

\item Suppose $A_{21}$ is eliminated row by row by means of $n$ steps of Gaussian elimination. Show that the bottom-right $(m-n)\times (m-n)$ block of the result is again the Schur complement.\\

Solution:\\

If $A_{21}$ is eliminated by GE, we must have some lower triangular matrix $L^{-1}$ multiplying the original matrix $A$ on the left. As the only rows that were eliminated were the last $m-n$ rows of $A$, we can express $L^{-1}$ as a block matrix:
$$L^{-1}=\left[\begin{array}{cc}I&\\C&I\end{array}\right].$$
Then if we multiply $A$ by $L^{-1}$ we should get some upper triangular block matrix with the first $n$ rows unchanged:
$$\left[\begin{array}{cc}I&\\C&I\end{array}\right]\left[\begin{array}{cc}A_{11}&A_{12}\\A_{21}&A_{22}\end{array}\right]=\left[\begin{array}{cc}A_{11}&A_{12}\\&S\end{array}\right].$$
We then solve for $C$:
$$CA_{11}+A_{21}=0\implies CA_{11}=-A_{21}\implies C=-A_{21}A_{11}^{-1}.$$
We can then define $S$:
$$CA_{12}+A_{22}=S\implies -A_{21}A_{11}^{-1}A_{12}+A_{22}=S.$$
Thus $S$ is the Schur complement, and we are done.
\end{enumerate}
\item The matrix $A\in\mathbb{C}^{m\times m}$ is diagonally dominant if
$$|a_{ii}|>\sum_{j=1,j\neq i}^m|a_{ij}|,\text{   }i=1,2,...,m.$$
\begin{enumerate}
\item Prove that if $A$ is diagonally dominant, then any principle submatrix of $A$ is diagonally dominant.\\
    
    Solution:\\
    
    Any principal submatrix of a diagonally dominant matrix $A$ has diagonal composed of only elements from the diagonal of $A$. Since the principal submatrix has fewer columns than the matrix $A$, the sum above is always going to be, for a principal submatrix, less than or equal to the sum for $A$. Thus any principal submatrix of $A$ must be diagonally dominant.\\


\item Prove that if $A$ is diagonally dominant, then $A$ is nonsingular.\\

Solution:\\

We use Gerschgorin disks to prove this conjecture. For $A$ to be nonsingular, it must have all nonzero eigenvalues. We note that each eigenvalue is located within a Gerschgorin disk defined
$$D_i=\{z|\text{  }|z-a_{ii}|\leq \sum_{j=1,j\neq i}^m|a_{ij}|\}.$$
We can expand the expression for the disk as such:
$$-\sum_{j=1,j\neq i}^m|a_{ij}|\leq z-a_{ii}\leq \sum_{j=1,j\neq i}^m|a_{ij}|$$
$$\implies a_{ii}-\sum_{j=1,j\neq i}^m|a_{ij}|\leq z\leq a_{ii}+\sum_{j=1,j\neq i}^m|a_{ij}|.$$
Since $A$ is diagonally dominant, $a_{ii}-\sum_{j=1,j\neq i}^m|a_{ij}|>0$ for $a_{ii}>0$ and $a_{ii}+\sum_{j=1,j\neq i}^m|a_{ij}|<0$ for $a_{ii}<0$. Therefore, there can be no zero eigenvalues and $A$ is nonsingular.\\


\item Prove that if $A$ is diagonally dominant then it will have an LU decomposition.\\

Solution:\\

Since each principal submatrix, including the upper left $k\times k$ block for each $k$, is diagonally dominant, each block is therefore nonsingular. Thus, by NLA 20.1 $A$ has an LU decomposition.
\end{enumerate}
\item lufactor code:
\begin{lstlisting}
% Gaussian Elimination with Partial Pivoting
function [L,U,P]=lufactor(A)
[~,m]=size(A);
U=A;
L=eye(m,m);
P=eye(m,m);
for k=1:(m-1)
    max=abs(U(k,k));
    maxindex=k;
    for i=k+1:m
        if abs(U(i,k))>max
            max=abs(U(i,k));
            maxindex=i;
        end
    end
    b=U(k,k:m);
    U(k,k:m)=U(maxindex,k:m);
    U(maxindex,k:m)=b;
    c=L(k,1:(k-1));
    L(k,1:(k-1))=L(maxindex,1:(k-1));
    L(maxindex,1:(k-1))=c;
    d=P(k,1:m);
    P(k,1:m)=P(maxindex,1:m);
    P(maxindex,1:m)=d;
    for j=k+1:m
        L(j,k)=U(j,k)/U(k,k);
        U(j,k:m)=U(j,k:m)-L(j,k)*U(k,k:m);
    end
end
\end{lstlisting}

lusolve code:
\begin{lstlisting}
function x=lusolve(b,L,U,P)
[~,m]=size(L);
 z=P*b;


 y=zeros(m,1);
 x=zeros(m,1);

 for i=1:m
     y(i)=(z(i)-dot(y,L(i,1:m)))/L(i,i);
 end

 for k=m:-1:1
     x(k)=(y(k)-dot(x,U(k,1:m)))/U(k,k);
 end
 \end{lstlisting}
\begin{enumerate} \item LU factorization of $A$.
\begin{lstlisting}>> A=[2,1,1,0;4,3,3,1;8,7,9,5;6,7,9,8]

A =

     2     1     1     0
     4     3     3     1
     8     7     9     5
     6     7     9     8

>> [L,U,P]=lufactor(A)

L =

    1.0000         0         0         0
    0.7500    1.0000         0         0
    0.5000   -0.2857    1.0000         0
    0.2500   -0.4286    0.3333    1.0000


U =

    8.0000    7.0000    9.0000    5.0000
         0    1.7500    2.2500    4.2500
         0         0   -0.8571   -0.2857
         0         0         0    0.6667


P =

     0     0     1     0
     0     0     0     1
     0     1     0     0
     1     0     0     0

>> norm(P*A-L*U,2)

ans =

   1.1102e-16

>> P*A

ans =

     8     7     9     5
     6     7     9     8
     4     3     3     1
     2     1     1     0

>> L*U

ans =

    8.0000    7.0000    9.0000    5.0000
    6.0000    7.0000    9.0000    8.0000
    4.0000    3.0000    3.0000    1.0000
    2.0000    1.0000    1.0000    0.0000
\end{lstlisting}
\item Using LU factorization to solve $Ax=b$.
\begin{lstlisting}
>> A

A =

     2     1     1     0
     4     3     3     1
     8     7     9     5
     6     7     9     8

>> b

b =

     7
    23
    69
    79

>> [L,U,P]=lufactor(A);
>> x=lusolve(b,L,U,P)

x =

    1.0000
    2.0000
    3.0000
    4.0000

>> A*x

ans =

    7.0000
   23.0000
   69.0000
   79.0000

>> norm(A*x-b,2)

ans =

   1.7764e-15
   \end{lstlisting}
   \end{enumerate}
\end{enumerate}
\end{document}