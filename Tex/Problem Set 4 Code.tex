\documentclass[12pt]{article}

\usepackage{graphics}
\usepackage{amsmath}
\usepackage{amsfonts}
\usepackage{amssymb}
\usepackage[table]{xcolor}



%\usepackage[active]{srcltx} % SRC Specials for DVI Searching

% Over-full v-boxes on even pages are due to the \v{c} in author's name
\vfuzz2pt % Don't report over-full v-boxes if over-edge is small

% THEOREM Environments ---------------------------------------------------

 \newtheorem{thm}{Theorem}[section]
 \newtheorem{cor}[thm]{Corollary}
 \newtheorem{lem}[thm]{Lemma}
 \newtheorem{prop}[thm]{Proposition}
 %\theoremstyle{definition}
 \newtheorem{defn}[thm]{Definition}
 %\theoremstyle{remark}
 \newtheorem{rem}[thm]{Remark}
 \numberwithin{equation}{section}
% MATH -------------------------------------------------------------------
 \DeclareMathOperator{\RE}{Re}
 \DeclareMathOperator{\IM}{Im}
 \DeclareMathOperator{\ess}{ess}
 \newcommand{\eps}{\varepsilon}
 \newcommand{\To}{\longrightarrow}
 \newcommand{\h}{\mathcal{H}}
 \newcommand{\s}{\mathcal{S}}
 \newcommand{\A}{\mathcal{A}}
 \newcommand{\J}{\mathcal{J}}
 \newcommand{\M}{\mathcal{M}}
 \newcommand{\W}{\mathcal{W}}
 \newcommand{\X}{\mathcal{X}}
 \newcommand{\BOP}{\mathbf{B}}
 \newcommand{\BH}{\mathbf{B}(\mathcal{H})}
 \newcommand{\KH}{\mathcal{K}(\mathcal{H})}
 \newcommand{\Real}{\mathbb{R}}
 \newcommand{\Complex}{\mathbb{C}}
 \newcommand{\Field}{\mathbb{F}}
 \newcommand{\RPlus}{\Real^{+}}
 \newcommand{\Polar}{\mathcal{P}_{\s}}
 \newcommand{\Poly}{\mathcal{P}(E)}
 \newcommand{\EssD}{\mathcal{D}}
 \newcommand{\Lom}{\mathcal{L}}
 \newcommand{\States}{\mathcal{T}}
 \newcommand{\abs}[1]{\left\vert#1\right\vert}
 \newcommand{\set}[1]{\left\{#1\right\}}
 \newcommand{\seq}[1]{\left<#1\right>}
 \newcommand{\norm}[1]{\left\Vert#1\right\Vert}
 \newcommand{\essnorm}[1]{\norm{#1}_{\ess}}
\usepackage{graphicx}
\usepackage{amsmath}
\usepackage{amsfonts}
\usepackage{amssymb}
%TCIDATA{OutputFilter=latex2.dll}
%TCIDATA{CSTFile=LaTeX article (bright).cst}
%TCIDATA{Created=Fri Nov 02 10:44:42 2001}
%TCIDATA{LastRevised=Mon Dec 10 11:56:49 2001}
%TCIDATA{<META NAME="GraphicsSave" CONTENT="32">}
%TCIDATA{<META NAME="DocumentShell" CONTENT="General\Blank Document">}
%TCIDATA{Language=American English}
\newtheorem{theorem}{Theorem}
\newtheorem{acknowledgment}[theorem]{Acknowledgment}
\newtheorem{algorithm}[theorem]{Algorithm}
\newtheorem{axiom}[theorem]{Axiom}
\newtheorem{case}[theorem]{Case}
\newtheorem{claim}[theorem]{Claim}
\newtheorem{conclusion}[theorem]{Conclusion}
\newtheorem{condition}[theorem]{Condition}
\newtheorem{conjecture}[theorem]{Conjecture}
\newtheorem{corollary}[theorem]{Corollary}
\newtheorem{criterion}[theorem]{Criterion}
\newtheorem{definition}[theorem]{Definition}
\newtheorem{example}[theorem]{Example}
\newtheorem{exercise}[theorem]{Exercise}
\newtheorem{lemma}[theorem]{Lemma}
\newtheorem{notation}[theorem]{Notation}
\newtheorem{problem}[theorem]{Problem}
\newtheorem{proposition}[theorem]{Proposition}
\newtheorem{remark}[theorem]{Remark}
\newtheorem{solution}[theorem]{Solution}
\newtheorem{summary}[theorem]{Summary}
\newenvironment{proof}[1][Proof]{\textbf{#1.} }{\ \rule{0.5em}{0.5em}}
\renewcommand\refname{}
\renewcommand\thefootnote{}
\textheight=9in \topmargin=-0.6in \everymath{\displaystyle}
\textwidth=6.5in \oddsidemargin=0.05in
\renewcommand\arraystretch{1.5}
\newenvironment{amatrix}[1]{%
  \left[\begin{array}{@{}*{#1}{c}|c@{}}
}{%
  \end{array}\right]
}
\includeonly{}
\usepackage{amsfonts}
\usepackage{amssymb}
\usepackage{eucal}
\usepackage{multicol}
\usepackage[bw]{mcode}
\usepackage{listings}
\everymath{\displaystyle}

\begin{document}

\begin{lstlisting}
function [W,R]=house(A)
%householder algorithm
[m,n]=size(A);

for k=1:n
    I=k:m;
    x=A(k:m,k);
    e=zeros(m-k+1,1);
    e(1)=1;
    if x(1)==0
        V(I,k)=norm(x,2)*e+x;
    else
        V(I,k)=sign(x(1))*norm(x,2)*e+x;
    end
    V(I,k)=V(I,k)/norm(V(I,k),2);
    A(k:m,k:n)=A(k:m,k:n)-2*V(I,k)*(V(I,k)'*A(k:m,k:n));
end
W=V;
R=A(1:n,1:n);
\end{lstlisting}

\begin{lstlisting}
function Q=formQ(W)

[m,n]=size(W);

for i=1:n
    x=zeros(m,1);
    x(i)=1;
    for k=n:-1:1
        x(k:m)=x(k:m)-2*W(k:m,k)*(W(k:m,k)'*x(k:m));

    end
    Q(1:m,i)=x;
end
\end{lstlisting}

\begin{lstlisting}
function [Vmn,b]=newVandermonde(m,n)
%create an mxn vandermonde matrix with
%input points uniformly spaced from 0 to 1
b=zeros(m,1);
t=zeros(m,1);
Vmn=zeros(m,n);
for i=1:m
    t(i)=i*.02;
    for j=1:n
        Vmn(i,j)=t(i)^(j-1);
    end
    b(i)=cos(4*t(i));
end
\end{lstlisting}
\pagebreak
\begin{lstlisting}
function [xa,xb,xc,xd,xe,xf,xg]=leastSquaresSolver(A,b)
[m,n]=size(A);
%Normal Equations

B=A'*A;
R=chol(B);
w=R'^(-1)*A'*b;
xa=R^(-1)*w;


%QR factorization and Classical Gram-Schmidt

[Qc,Rc]=clgs(A);
xb=Rc^(-1)*Qc'*b;

%QR factorization and Modified Gram-Schmidt

[Qm,Rm]=mgs(A);
xc=Rm^(-1)*Qm'*b;

%QR factorization and Householder triangularization

[Qh,Rh]=house(A);
xd=Rh^(-1)*Qh'*b;

%QR factorization and Matlab qr

[Q,R]=qr(A,0);
xe=R^(-1)*Q'*b;

%Matlab backslash

xf=A\b;

%Matlab SVD

[U,S,V]=svd(A,0);
y=S^(-1)*U'*b;
xg=V*y;

fprintf('    Normal                CLGS                    MGS                   HOUSE\n');
for i=1:n
    fprintf(' %22.15e %22.15e %22.15e %22.15e\n', xa(i),xb(i),xc(i),xd(i))
end;

fprintf('    Matlab QR             Matlab backslash        SVD\n');
for i=1:n
    fprintf(' %22.15e %22.15e %22.15e\n',xe(i),xf(i),xg(i))
end;
\end{lstlisting}
\end{document}
